% !TeX root = document.tex
% !TeX encoding = UTF-8 Unicode

\section{Introduction}%
\label{Introduction}

The objective of this report to show the findings of simulations on application of PID control system on a bicycle model framework.
The differential formulas used to describe and simulate the system are as follows: 
\begin{align}
    \dot x =& v\cos \theta 
    \\
    \dot y =& v\sin \theta 
    \\
    \dot \theta =& \frac{v}{L}\cos u 
\end{align}
Where x is horisontal position, y is vertical position, theta is angle of vehicle, u is is angle of steering wheel and v is velocity. The derived differential equation is as follows:
\begin{equation}
    \dot z =& f(z, u)
\end{equation}
Throughout the simulations, consistent initial state values are going to be used which are: 
\begin{align}
     y =& 0.3m \mbox{ - lateral position(m)}
    \\
    x =& 0m \mbox{ - horisontal position(m)}
    \\
    \theta =& 5 \mbox{ - angle of vehicle(deg)}
\end{align}
\pagebreak
\\
Python is used as the programming language to carry out the simulation due to accessability of use, ease of installing and wide community support for scientific computing with libraries such as NumPy. Development of code and simulation done within a managed package of scientific computing, Anaconda. Matplotlib, NumPy, and SciPy libraries were used to complete this task.

Completed and functioning code can be found \href{https://gist.github.com/Vensim/3dc94b8bcbc3232b5db5f21bf90e3449}{here}. Function and class included within the file, no additional repositories required.
